\documentclass[twocolumn]{jarticle} % 日本語の場合
%\documentclass[twocolumn]{article} % for English manuscript

%%% 多値論理研究会用マクロの使い方
%%% \usepackage[mvlforum]{mymvlmacros} %(多値論理フォーラムの場合)または
%%% \usepackage[mvlnishuken]{mymvlmacros} %(多値論理二種研の場合)
\usepackage[mvlnishuken]{mymvlmacros}

%--------多値論理フォーラムのパラメタ(多値論理二種研の場合は不要)--------
\renewcommand{\mvlvolume}{39} % 多値論理フォーラムの巻号
\renewcommand{\mvlpresennumber}{1} % 多値論理フォーラムの発表番号
\renewcommand{\mvlyear}{2016} % 年
\renewcommand{\mvlmonth}{9} % 月

%%%%%%%%%%%%%%%%%%%%%%%%%%%%%%%%%%%%%%%%%%%%%%%%%%%%%%%%%
%% <local definitions here>

\usepackage{latexsym}
\usepackage{graphicx}
\usepackage{amssymb,amsmath}
%\usepackage{times,mathptm}
% ギリシャ文字を含めた欧文フォントがTimesになる。
\DeclareSymbolFont{letters}{OML}{ztmcm}{m}{it}

%% </local definitions here>
%%%%%%%%%%%%%%%%%%%%%%%%%%%%%%%%%%%%%%%%%%%%%%%%%%%%%%%%%

\title{
{\LARGE\textbf{盛岡三大麺に関する私見}}\\
{\Large\textbf{A Note on the Three Great Noodles of Morioka}}
}

\author{
 \mydoublename{平山貴司$^1$}{Takashi Hirayama} 
 \mydoublename{岩手太郎$^1$}{Taro Iwate}
 \mydoublename{盛岡花子$^2$}{Hanako Morioka}
\medskip\\
 \mydoublename{$^1$〇〇大学 〇〇学部 〇〇学科}{$^1$Faculty of Something,  Somewhere University}\\
 \mydoublename{$^2$〇〇研究所 〇〇部}{$^2$Department of Something, Somewhere Rearch Center}\\
 {E-mail: \{\texttt{hira@bbb.cc.dd}, \texttt{iwa@ccc.ddd.ee}, \texttt{mori@dd.eee.fff}\}}
}

%% 日本語の場合は,オプション引数の「Abstract」は取ること。
%% 例: 
%% \abstract{
%% これは日本語の概要のサンプルである。
%% これは日本語の概要のサンプルである。
%% これは日本語の概要のサンプルである。
%% }
\abstract[Abstract]{
In Morioka cuisine, Reimen, Jajamen, and Wankosoba are especially famous as
the three great noodles.
This manuscript describes how the local people in the city love these noodles.
Advice for tourists is also given.
}

\begin{document}
\maketitle

%%%%%%%%%%%%%%%%%%%%%%%%%%%%%%%%%%%%%%%%%%%%%%%%%%%%%%%%%%%%%%%%
\section{盛岡三大麺}
盛岡市民は、\textbf{盛岡冷麺}、\textbf{盛岡じゃじゃ麺}、\textbf{わんこそば}を\textbf{盛岡三大麺}と呼び、
親しんでいる\cite{IwaNic2014,Kik2015}。
マスコミでも取り上げられてきたおかげで、それなりに有名になり、
観光客にも人気がある。
人気に伴い、スケールを拡大して\textbf{いわて三大麺}と称することも見受けられるようになっている。
%%%%%%%%%%%%%%%%%%%%%%%%%%%%%%%%%%%%%%%%%%%%%%%%%%%%%%%%%%%%%%%%

%%%%%%%%%%%%%%%%%%%%%%%%%%%%%%%%%%%%%%%%%%%%%%%%%%%%%%%%%%%%%%%%
\section{盛岡冷麺}
%%%%%%%%%%%%%%%%%%%%%%%%%%%%%%%%%%%%%%%%%%%%%%%%%%%%%%%%%%%%%%%%
驚くほどコシの強い麺とキムチの辛味が効いたスープの冷製麺料理。
朝鮮半島由来の冷麺が盛岡で独自の進化を遂げて、盛岡冷麺となった。
盛岡では単に「冷麺」と呼ぶことが普通である。
実は盛岡には冷麺専門店はほとんどなく、
盛岡市民が冷麺を食べに行くところは焼肉店。
店では、辛味の強さを数段階から選べるが、
辛いのが苦手な人や初めての人は\textbf{別辛}(べつから)を選ぶと良い。
別辛は、辛味を冷麺とは別の容器で提供することである。
自分で辛味を少しずつ冷麺に入れて様子をみながら食べることができるので、
辛味による失敗を防ぐのに有効である。

%%%%%%%%%%%%%%%%%%%%%%%%%%%%%%%%%%%%%%%%%%%%%%%%%%%%%%%%%%%%%%%%
\section{盛岡じゃじゃ麺}
%%%%%%%%%%%%%%%%%%%%%%%%%%%%%%%%%%%%%%%%%%%%%%%%%%%%%%%%%%%%%%%%
うどんのような麺の上に肉味噌が乗っている汁なし麺料理。
中国の炸醤麺(ジャージャンメン)が
盛岡で独自の進化を遂げて、盛岡じゃじゃ麺となった。
盛岡では単に「じゃじゃ麺」と呼ぶことが普通である。
表記においては、ひらがなの「じゃじゃ」に漢字の「麺」の組み合わせで書くことが多い。
麺をよく混ぜて、肉味噌をまんべんなく全体に行き渡らせてから食べるのが作法である。
食後の\textbf{鶏蛋湯}(チータンタン)という玉子スープも大事な楽しみである。
あえて肉味噌を少しだけ残して食べ終わった皿に、
客自身が生卵を割り入れてかき混ぜ、
その皿を店員に渡すと、鶏蛋湯になって戻ってくる。
予備知識のない観光客にとっては、楽しみ方の作法がわかりにくい点が多いので、
素直に「観光客なので食べ方を教えてください。」と言って、
店員に作法を教えてもらうのが得策。

%%%%%%%%%%%%%%%%%%%%%%%%%%%%%%%%%%%%%%%%%%%%%%%%%%%%%%%%%%%%%%%%
\section{わんこそば}
%%%%%%%%%%%%%%%%%%%%%%%%%%%%%%%%%%%%%%%%%%%%%%%%%%%%%%%%%%%%%%%%
おわんに入った一口大のそばを、客が食べるたびに給仕が補給するという掛け合い
が楽しいそば料理。
わんこそばの料金は定額制食べ放題としている店が多い。
観光客へのもてなし料理という位置づけであるため、盛岡市民自身が
わんこそばを食べる機会は、冷麺、じゃじゃ麺と比較すると少なめである。
例年わんこそばの大会が開かれているため、大食い・早食いに挑戦する
イメージが強いが、
本来もてなし料理であり、観光客には和気あいあいと食べて楽しんでもらいたい。


%%%%%%%%%%%%%%%%%%%%%%%%%%%%%%%%%%%%%%%%%%%%%%%%%%%%%%%%%%%%%%%%
% References
%%%%%%%%%%%%%%%%%%%%%%%%%%%%%%%%%%%%%%%%%%%%%%%%%%%%%%%%%%%%%%%%

\begin{thebibliography}{9}% more than 9 --> 99 / less than 10 --> 9

\bibitem{IwaNic2014}
いわなみつ, 二丁目のママ, ``岩手あるある'', TOブックス, 2014年12月

\bibitem{Kik2015}
菊池幸見, ``岩手共和国のオキテ100ヵ条'', メイツ出版, 2015年6月
\end{thebibliography}

\end{document}
